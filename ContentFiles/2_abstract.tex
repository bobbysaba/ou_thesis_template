% format the chapter/section details
\chapter*{Abstract}
\chaptermark{Abstract}
\label{chap:Abstract}
\addcontentsline{toc}{chapter}{Abstract}

% insert your text below the line
%-------------------------------------------
While immense enhancements to meteorological observations and simulations have been made over the last several decades, two lingering questions continue to plague the community: (1) Why do some storms produce tornadoes while others do not, and (2) why do storms in seemingly identical environments go on to produce different hazards? Developing robust answers to these questions are critical as tornadoes continue to be a top cause for weather-related fatalities. Storm environment research through simulated storms, reanalysis products, and limited observations have guided the current understanding and have identified differences between environments supportive of tornadic and non-tornadic supercells. This work aims to use more detailed observations through Doppler wind lidars to capture the storm environment evolution in both time and space. 

Since 2016, the National Severe Storms Laboratory has operated mobile scanning Doppler wind lidars within the inflow of supercell thunderstorms as part of collaborative field projects (mini-MPEX, TORUS, PERiLS, TORUS-LiTE, and LIFT). The research goal of this instrument is to observe the evolution of storm inflow properties to gain a better understanding of the storm environment surrounding an evolving supercell. Various scanning and post-processing techniques to capture the wind profile evolution via DWL observations have been used. Prior to 2022, a velocity azimuth display technique was used to retrieve a vertical wind profile every 3-5 minutes using 8 points 20 degrees off zenith.  In 2022-23, a continuous scanning mode was implemented to retrieve wind profiles as frequent as every 5 seconds. Both of these scanning strategies yielded vertical profiles of derived horizontal winds about every 18 meters with the first usable data point around 75 meters AGL after filtering based on a signal-to-noise ratio threshold. A new optimal estimation technique was designed to incorporate co-located rawinsonde and surface observations into the wind retrievals that provide data between the surface and 75 meters. These observations are used to validate the post processing techniques and output will be compared to previously used methods.

Focus will be placed on the evolution of the near-ground kinematic characteristics and the storm induced environmental modification in both tornadic and non-tornadic storms. To remove background trends from the environmental evolution, Rapid Refresh model-based analysis profiles are used. Full, surface-based wind profiles will allow for the quantification and time evolution of severe weather forecasting parameters, such as storm-relative helicity, storm-relative winds, etc. Implications of these results on recent studies discussing the relative importance of streamwise vorticity versus storm-relative winds on supercell properties and tornado production will be discussed. 

Results show that differences in mesocyclone intensity between tornadic and non-tornadic supercells cause the downstream spatial variability of their environments. This is particularly true when quantifying storm-induced accelerations, which can lead to the length scales at which mesocyclone induced perturbations can be observed. Furthermore, low-level ground-relative winds tend to provide greater differences in the storm environment whereas storm-relative winds contain greater composite differences above 1 kilometer. Also explored are the vorticity trends in these storm environments in time and space, and the horizontal vorticity components (and associated differences) will be shown. 





