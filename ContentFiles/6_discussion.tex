% format the chapter/section details
\chapter{Summary and Conclusions}
\label{chap:discussion}

% insert your text below the line
%-------------------------------------------
To this point in severe storms research, the use of DWL data in convective environments has been minimal. This work provides an atypical use case for the instrument and displays the benefits of using an instrument of this nature to observe severe convective storm environments, particularly in detailing the near-ground wind profiles. Although the main drawback of using DWLs for this purpose lies in their blind spot below  about 60 to 80 meters, improvements to post-processing techniques have aided in alleviating this problem. In this study, we have explored the use of DWL wind retrievals and a wind estimation method (WINDoe) to quantify the evolution and variability of the low-level wind profile in storm environments. Specifically using the prior dataset created for use in this analysis, WINDoe provides an opportunity to combine DWL wind retrievals with other observational data sets to create a unified analysis of the wind profile over the lower troposphere in severe storm environments. Furthermore, WINDoe quantifies uncertainty in the retrievals, something missing from many observational based studies to this point in the science. Overall, the ability for this instrument to be nimble in operational settings, tied together with strides in data-analysis techniques should highlight the possibilities of utilizing DWLs in severe storms research.

This research provides an in-depth observation-based analysis into how supercell storm environment wind profiles evolve in time and space exploiting the ability of DWLs to retrieve wind profiles as frequently as every 5-10 seconds with high vertical resolution (about 18 meters). This includes the environmental differences between tornadic and non-tornadic storm environments, building on a lengthy literature of storm environment research rooted in observations, reanalysis, and simulated storms. Many previous efforts have yielded conflicting results, so this work seeks to provide clarity through the use of real-world datasets while targeting the most intense non-tornadic storms, findings that could further inform the forecasting community. Based on previous efforts this work addressed the following hypotheses:

\begin{enumerate}
    \item Kinematics vary more in non-tornadic storms in space, with drastic changes in SRH from the far-field to near-field not seen in tornadic composites \citep{parker2014composite}. 
    \item Low-level, inflow winds in non-tornadic supercell environments contain more crosswise vorticity compared to their tornadic supercell counterparts \citep{parker2014composite}.
    \item SRW plays an insignificant role in tornado dynamics and thus, there are minimal differences in low-level SRW profiles in tornadic and non-tornadic environments \citep{peters2023disentangling}.
    \item Streamwise vorticity drives changes in SRH in the storm environment and provides the skill between SRH and tornado prediction \citep{peters2023disentangling}.
    \item Low-level parameters such as SRH$_{500m}$ provide better discriminatory skill compared to deeper-layer parameters such as SRH$_{1km}$ or SRH$_{3km}$ \citep{coffer2019using, coniglio2020insights}.
\end{enumerate}